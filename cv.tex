\documentclass{tccv}
\usepackage[utf8]{inputenc}
\usepackage[T1]{fontenc}
\usepackage[frenchb,english]{babel}
\usepackage{graphicx}
\usepackage{ulem}


\renewcommand\personal[5][]{% 4 à l'origine
    \needspace{0.5\textheight}%
    \newdimen\boxwidth%
    \boxwidth=\dimexpr\linewidth-2\fboxsep\relax%
    \newlength\tailleImage% hauteur de l'image
    \settoheight\tailleImage{\includegraphics[width=0.25\boxwidth]{#5}}% 25%, valeur modifiable
    \newlength\ex% hauteur de la lettre "a"
    \settoheight\ex{a}%
    \newlength\declg% décalage
    \setlength\declg{0.5\tailleImage}%
    \addtolength\declg{-0.5\ex}%
    \colorbox[HTML]{F5DD9D}{%
    \begin{tabularx}{0.75\boxwidth}{c|X}% 75%, valeur modifiable
    \Writinghand & {#2} \smallskip\\
    \Telefon     & {#3} \smallskip\\
    \Letter      & \href{mailto:#4}{#4}
    \ifstrempty{#1}{}{\smallskip\\ \Lightning & \href{http://#1}{#1}}
    \end{tabularx}%
    \raisebox{-\declg}{\includegraphics[width=0.25\boxwidth]{#5}}% 25%, valeur modifiable
    }}

\begin{document}

\part{Mathias Le Borgne}

\section{Expérience professionelle}

\begin{eventlist}

\item{Septembre 2014 -- Présent}
     {Aldebaran Robotics, Paris}
     {Product Owner}

Design de l'interaction homme-robot (vocale et gestuelle) 
encadrement d'une équipe de développement agile de 8 personnnes.

Définition itérative des spécifications avec nos clients,
participation à l'implémentation 
et suivi du projet jusqu'à la livraison.

\item{Septembre 2012 -- Septembre 2014}
     {Aldebaran Robotics, Paris}
     {Développeur interaction}

Développement (C++/python)
d'un algorithme de contact visuel 
pour les robots Pepper et Nao, 
qui a donné lieu à un brevet (EP2933064).

Assemblage de briques logicielles de perception
(tracking visuel, détection de son, mouvement, toucher) 
et de mouvements du robot
pour détecter l'utilisateur et le suivre du regard.

Collaboration avec l'équipe "Expérience Utilisateur".

\item{Mars 2012 -- Août 2012}
     {IRCAM, Paris}
     {Stage de Recherche}

Étude d'un système non linéaire de production vocale piloté par un
modèle d’aire glottique pour la synthèse de la parole.


\item{Août 2010}
     {Xiang Sheng Packing, Shenzhen, Chine}
     {Stage ouvrier}

Un mois de travail dans une usine de conditionnement.


\section{Projets}

\begin{eventlist}

\item{juin 2015}
     {Projet Personnel}
     {Site internet}

\textnormal{\url{www.steleforest.com}}, site réalisé avec Django, affichage bilingue de textes de la littérature chinoise. 


\item{Septembre 2011 -- Mars 2012}
    {"Le Laboratoire", Paris}
    {Projet d'innovation}
    
Projet avec des étudiants en design :
réflexion autour du packaging de l'eau, a conduit à un dépôt de brevet pour un bouchon innovant.


\end{eventlist}

\personal
    {197 avenue Pierre Brossolette 
     \newline 92120 Montrouge}
    {06 23 30 93 79}
    {mathias.leborgne@gmail.com}
    {selfie} % nom du fichier de la photo (son extension est facultative)


\section{Formation}

\begin{yearlist}

\item[Diplome d'ingénieur]{2009--2012}
     {Télécom ParisTech, Paris}
     {Spécialisation en traitement du signal}

\item[Master ATIAM]{2011--2012}
    {IRCAM-UPMC, Paris}
    {Acoustique, traitement du signal, 
     informatique appliqués à la musique}

\item[Classes Préparatoires]{2006--2009}
    {Lycée Henri Wallon, Valenciennes (Nord)}
    {MPSI et MP*}

\end{yearlist}

\section{Compétences}

\begin{factlist}

\item{Langages}
     {C++, Python, Matlab}

\item{Outils}
     {Git, Django, Boost, QtCreator, SublimeText, \LaTeX, Max/MSP}

\item{Autres}
     {Développement agile \\
      (méthode SCRUM), \\
      Interaction homme-robot,
      Programmation concurrente,
      Machine learning}

\end{factlist}

\section{Langues}

\begin{factlist}

\item{Anglais}
    {Courant}
    %{Courant : plusieurs séjours en immersion dans l'Oregon, un chef américain pendant un an}
\item{Allemand}
    {Bon niveau}
    %{Bon niveau : un séjour en immersion à Stuttgart}
\item{Chinois}
    {Notions}
    %{Notions : deux ans d'études et un séjour de deux mois en Chine
    %(dont un mois à Beijing Language and Culture
    %University)}

\end{factlist}

\section{Hobbies}
% Interests en anglais...

\begin{factlist}

\item{Musique}
     {Guitare classique}

\item{Art}
     {Dessin d'art, intérêt pour la peinture}

\end{factlist}

\end{document}
